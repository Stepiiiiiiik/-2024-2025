\documentclass{article}
\usepackage[utf8]{inputenc}
\usepackage[russian]{babel}
\usepackage{epigraph}
\usepackage{amsfonts}
\usepackage{mathtools}
\usepackage{amsmath}

\begin{document}
	\begin{obeylines}
	\begin{titlepage}
		 \vspace*{\stretch{1.0}}
		\begin{center}
			\Huge\textbf{Алгебра}\\
			\bigskip
			\bigskip
			\Large\textbf{Конспект лекций В. В. Нестерова, 2024}
		\end{center}
		\vspace*{\stretch{2.0}}
	\end{titlepage}
	
	\epigraph{Пётр живёт в пунке
		И вот пошел первый год,
		Пётр ушёл от людей,
		Он ушёл от мирских хлопот,
		Он просто устал от жизни
		И не держит зла на людей,
		Но было время –
		Он был носителем великих идей
		Теперь матмех стал его домом,
		Он здесь может спокойно ботать,
		Он компилирует Си в голове
		И его решения никак не взломать
		
		А по ночам он приходит ко мне,
		Он зовёт меня в коворк,
		Он идёт к луне,
		Он видит ночь, как никто другой…}{\textbf{раз два три четыре пять, \\ с рифмой с детства я дружу}}
	\section*{Отношение эквивалентности и разбиения}	
	Начнем с примера. Работаем с $\mathbb{Z}$, зафиксируем $m \in \mathbb{Z}, m > 0, x \sim y \iff x - y \equiv 0\pmod{m}.$
	\bigskip
	
	Проверка:
	1) $x \sim x,$ поскольку $x - x \equiv 0\pmod{m}$
	2) $x \sim t \rightarrow x - y \equiv 0\pmod{m} \rightarrow y - x \equiv 0\pmod{m} \rightarrow y \sim x$
	3) $x - y \equiv 0\pmod{m}, y - z \equiv 0\pmod{m} \rightarrow (x - y) + (y - z) = x - z \equiv 0\pmod{m}$
	Заданное нами отношение действительно является отношением эквивалентности.
	
	$[0] \coloneqq \{0, m, -m, 2m, -2m, ...\}$
	$[1] \coloneqq \{1, m + 1, -m + 1, ...\}$
	$\vdots$
	$[a] \coloneqq \{a, m + a, -m + a, 2m + a, -2m + a, ...\}$
	$a = 0, ..., m - 1$
	[a] называется классом эквивалентности
	\underline{\textbf{Теорема.}}
	1) $\sim$ задает на X разбиение на классы эквивалентности
	2) Разбиение множества X задаёт на X отношение эквивалентности
	\textbf{Д-во:}
	1) $x \in X, X_{i} \coloneqq \{y \in X | x \sim y\}$
	Покажем, что $\{X_{i}\}_{i \in I}$ является разбиением X. Очевидно, что объединение этого семейства равно X. Проверим, что классы эквивалентности не могут пересекаться.
	Действительно, предположим противное: пусть $x \in X, x \in X_{i} = [y], x \in X_{j} = [z], i, j \in I, i \neq j.$ Воспользуемся транзитивностью эквивалентности: $x \sim y, x \sim z \Rightarrow y \sim z \Rightarrow$ [y] и [z] совпадают $\Rightarrow$ противоречие - мы брали два разных класса эквивалентности.
	2) $\{X\}_{i \in I}$ - разбиение X. Введем следующее отношение - $x \sim y \iff \exists i \in I:  x \in X_{i}  \wedge  y \in X_{i}.$
	Проверим:
	1) рефлексивность очевидна
	2) $x, y \in X_{i} \Rightarrow y, x \in X_{i} \Rightarrow y \sim x$
	3) $x, y \in X_{i} \wedge y, z \in X_{i} \Rightarrow x \in X_{i} \wedge z \in X_{i} \Rightarrow x \sim z$
	
	
	\section*{Перестановки и определение группы}
	\underline{\textbf{Опр.}} Биективное отображение конечного множества $\sigma:  X \rightarrow X$ называется \textbf{перестановкой}.
	Записать перестановку можно следующим образом:
	$\sigma = \begin{pmatrix} 1 & 2 & 3 & ... & n \\ i_{1} & i_{2} & i_{3} & ... & i_{n} \end{pmatrix}$
	\underline{\textbf{Опр.}} \textbf{Группой} называется множество G с заданной на нем бинарной операцией $\circ$ со следующими свойствами:
		1) ассоциативность операции: $\forall x, y, z \in G: (x \circ y) \circ z = x \circ (y \circ z)$
		2) существование нейтрального элемента $e \in G$ такого, что: $\forall x \in G \\ x \circ e = e \circ x = x$. Легко заметить, что нейтральный элемент единственен.
		3) существование обратного элемента: $\forall x \in G \ \: \exists x^{-1}\in G: x \circ x^{-1} = x^{-1} \circ x = e$
		
	Теперь вернемся к перестановкам. Заметим, что мы можем перемножить две перестановки одного множества X - это просто композиция двух отображений. Продемонстрируем на примере:
	$ \sigma : X \rightarrow X, \tau : X \rightarrow X, \sigma = \begin{pmatrix} 1 & 2 & 3 & 4 \\ 2 & 1 & 4 & 3 \end{pmatrix}, \tau = \begin{pmatrix} 1 & 2 & 3 & 4 \\ 4 & 3 & 2 & 1 \end{pmatrix}$
	$\sigma \circ \tau = \begin{pmatrix} 1 & 2 & 3 & 4 \\ 2 & 1 & 4 & 3 \end{pmatrix} \begin{pmatrix} 1 & 2 & 3 & 4 \\ 4 & 3 & 2 & 1 \end{pmatrix} = \begin{pmatrix} 1 & 2 & 3 & 4 \\ 3 & 4 & 1 & 2 \end{pmatrix}$
	Заметим, что с таким умножением перестановки образуют группу, называющуюся $S_{n}$. Действительно, ассоциативность следует из ассоциативности композиции отображений, нейтральным элементом выступает тождественная перестановка $id$ (или $e$), и для каждой перестановки можем явно указать обратную ей. Пусть $ \sigma = \begin{pmatrix} 1 & 2 & 3 & ... & n \\ i_{1} & i_{2} & i_{3} & ... & i_{n} \end{pmatrix}$, тогда $\sigma^{-1} = \begin{pmatrix} i_{1} & i_{2} & i_{3} & ... & i_{n} \\ 1 & 2 & 3 & ... & n \end{pmatrix}$, можно легко проверить, что $\sigma^{-1} \circ \sigma = e$.
	
	\bigskip
	\underline{\textbf{Лемма.}} (вспомогательное утверждение, полезное не само по себе, а для доказательства других утверждений) $f:  X \rightarrow X$, f - биекция  $\iff  \exists f^{-1}.$
	Доказательство остается читателю как несложное упражнение.
	\end{obeylines}
\end{document}
