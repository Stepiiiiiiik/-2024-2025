\documentclass{article}
\usepackage[utf8]{inputenc}
\usepackage[russian]{babel}
\usepackage{epigraph}
\usepackage{amsfonts}
\usepackage{mathtools}
\usepackage{amsmath}
\usepackage{amsthm}
\usepackage{amssymb}
\newcommand{\existNoBar}{\mathord{\setbox0=\hbox{$\exists$}%
             \hbox{\kern 0.125\wd0%
                   \vbox to \ht0{%
                      \hrule width 0.75\wd0%
                      \vfill%
                      \hrule width 0.75\wd0}%
                   \vrule height \ht0%
                   \kern 0.125\wd0}%
           }}

\begin{document}
	\begin{obeylines}
	\begin{titlepage}
		 \vspace*{\stretch{1.0}}
		\begin{center}
			\Huge\textbf{Алгебра}\\
			\bigskip
			\bigskip
			\Large\textbf{Конспект лекций В. В. Нестерова, 2024}
		\end{center}
		\vspace*{\stretch{2.0}}
	\end{titlepage}
	
	\epigraph{Пётр живёт в пунке
		И вот пошел первый год,
		Пётр ушёл от людей,
		Он ушёл от мирских хлопот,
		Он просто устал от жизни
		И не держит зла на людей,
		Но было время –
		Он был носителем великих идей
		Теперь матмех стал его домом,
		Он здесь может спокойно ботать,
		Он компилирует Си в голове
		И его решения никак не взломать
		
		А по ночам он приходит ко мне,
		Он зовёт меня в коворк,
		Он идёт к луне,
		Он видит ночь, как никто другой…}{\textbf{раз два три четыре пять, \\ с рифмой с детства я дружу}}
	\section*{Отношение эквивалентности и разбиения}	
	Начнем с примера. Работаем с $\mathbb{Z}$, зафиксируем $m \in \mathbb{Z}, m > 0, x \sim y \iff x - y \equiv 0\pmod{m}.$
	\bigskip
	
	Проверка:
	1) $x \sim x,$ поскольку $x - x \equiv 0\pmod{m}$
	2) $x \sim t \rightarrow x - y \equiv 0\pmod{m} \rightarrow y - x \equiv 0\pmod{m} \rightarrow y \sim x$
	3) $x - y \equiv 0\pmod{m}, y - z \equiv 0\pmod{m} \rightarrow (x - y) + (y - z) = x - z \equiv 0\pmod{m}$
	Заданное нами отношение действительно является отношением эквивалентности.
	
	$[0] \coloneqq \{0, m, -m, 2m, -2m, ...\}$
	$[1] \coloneqq \{1, m + 1, -m + 1, ...\}$
	$\vdots$
	$[a] \coloneqq \{a, m + a, -m + a, 2m + a, -2m + a, ...\}$
	$a = 0, ..., m - 1$
	[a] называется классом эквивалентности
	\underline{\textbf{Теорема.}}
	1) $\sim$ задает на X разбиение на классы эквивалентности
	2) Разбиение множества X задаёт на X отношение эквивалентности
        \newpage
	\textbf{Доказательство:}
	1) $x \in X, X_{i} \coloneqq \{y \in X | x \sim y\}$
	Покажем, что $\{X_{i}\}_{i \in I}$ является разбиением X. Очевидно, что объединение этого семейства равно X. Проверим, что классы эквивалентности не могут пересекаться.
	Действительно, предположим противное: пусть $x \in X, x \in X_{i} = [y], x \in X_{j} = [z], i, j \in I, i \neq j.$ Воспользуемся транзитивностью эквивалентности: $x \sim y, x \sim z \Rightarrow y \sim z \Rightarrow$ [y] и [z] совпадают $\Rightarrow$ противоречие - мы брали два разных класса эквивалентности.
	2) $\{X\}_{i \in I}$ - разбиение X. Введем следующее отношение - $x \sim y \iff \exists i \in I:  x \in X_{i}  \wedge  y \in X_{i}.$
	Проверим:
	1) рефлексивность очевидна
	2) $x, y \in X_{i} \Rightarrow y, x \in X_{i} \Rightarrow y \sim x$
	3) $x, y \in X_{i} \wedge y, z \in X_{i} \Rightarrow x \in X_{i} \wedge z \in X_{i} \Rightarrow x \sim z$ \qed
	
	
	\section*{Перестановки и определение группы}
	\underline{\textbf{Опр.}} Биективное отображение конечного множества $\sigma:  X \rightarrow X$ называется \textbf{перестановкой}.
	Записать перестановку можно следующим образом:
	$\sigma = \begin{pmatrix} 1 & 2 & 3 & ... & n \\ i_{1} & i_{2} & i_{3} & ... & i_{n} \end{pmatrix}$
	\underline{\textbf{Опр.}} \textbf{Группой} называется множество G с заданной на нем бинарной операцией $\circ$ со следующими свойствами:
		1) ассоциативность операции: $\forall x, y, z \in G: (x \circ y) \circ z = x \circ (y \circ z)$
		2) существование нейтрального элемента $e \in G$ такого, что: $\forall x \in G \\ x \circ e = e \circ x = x$. Легко заметить, что нейтральный элемент единственен.
		3) существование обратного элемента: $\forall x \in G \ \: \exists x^{-1}\in G: x \circ x^{-1} = x^{-1} \circ x = e$
		
	Теперь вернемся к перестановкам. Заметим, что мы можем перемножить две перестановки одного множества X - это просто композиция двух отображений. Продемонстрируем на примере:
	$ \sigma : X \rightarrow X, \tau : X \rightarrow X, \sigma = \begin{pmatrix} 1 & 2 & 3 & 4 \\ 2 & 1 & 4 & 3 \end{pmatrix}, \tau = \begin{pmatrix} 1 & 2 & 3 & 4 \\ 4 & 3 & 2 & 1 \end{pmatrix}$
	$\sigma \circ \tau = \begin{pmatrix} 1 & 2 & 3 & 4 \\ 2 & 1 & 4 & 3 \end{pmatrix} \begin{pmatrix} 1 & 2 & 3 & 4 \\ 4 & 3 & 2 & 1 \end{pmatrix} = \begin{pmatrix} 1 & 2 & 3 & 4 \\ 3 & 4 & 1 & 2 \end{pmatrix}$
	Заметим, что с таким умножением перестановки образуют группу, называющуюся $S_{n}$. Действительно, ассоциативность следует из ассоциативности композиции отображений, нейтральным элементом выступает тождественная перестановка $id$ (или $e$), и для каждой перестановки можем явно указать обратную ей. Пусть $ \sigma = \begin{pmatrix} 1 & 2 & 3 & ... & n \\ i_{1} & i_{2} & i_{3} & ... & i_{n} \end{pmatrix}$, тогда $\sigma^{-1} = \begin{pmatrix} i_{1} & i_{2} & i_{3} & ... & i_{n} \\ 1 & 2 & 3 & ... & n \end{pmatrix}$, можно легко проверить, что $\sigma^{-1} \circ \sigma = e$.
	
	\bigskip
	\underline{\textbf{Лемма.}} (вспомогательное утверждение, полезное не само по себе, а для доказательства других утверждений) $f:  X \rightarrow X$, $f$ - биекция  $\iff  \exists f^{-1}.$
    \vspace{1em}
    \textbf{Доказательство:} остается читателю как несложное упражнение. \qed\\
    
    \underline{\textbf{Опр.}} Перестановка $\sigma$, действующая на $k$ элементов, называется \textbf{циклом длины $k$}, если: \\
    \begin{center}
           $ \sigma$ = \begin{pmatrix}
    i_1 & i_2 & ... & i_k \\ i_2 & i_3 & ... & i_1 
    \end{pmatrix}
    \end{center}

    \underline{\textbf{Теорема.}} \sigma \in $S_n$ \Longrightarrow $\sigma $ раскладывается в пр-е независимых циклов:
    \begin{center}
        $\sigma = \sigma_1 \sigma_2 \sigma_3 ...$\\
    \end{center}
    \textbf{Доказательство:} \existNoBar \  $\math{X} = \{ \math{1}, \math{2}, \ldots, \math{n}\ \}$  \quad $i,j$ \in \ $\math{X}$ 
    Введём отношение эквивалетности:
    $i \sim j$ \Longleftrightarrow \  $\exists k \geq 0$ \quad  $\sigma^k(i) = j$
    1) В набора: $\{i, \sigma(i), \sigma^2(i), ... \}$ \ $\exists k: \sigma^k(i) = i$. 
    \existNoBar \ $\sigma^s(i) = \sigma^{s+1}(i)$ \Longrightarrow \ $\sigma^{-s}(i) = \sigma^{s+1}(i) = \sigma^{-s}\sigma^s(i) \Longrightarrow \sigma^k(i) = i$. Если это не так, значит все последовательные степени различны. Однако множество конечное, поэтому в некоторый момент $\sigma^k(i) = i$
    
    2) Если $i\sim j \Longrightarrow \sigma^k(i) = j \ \Longrightarrow i = \sigma^{-k}(j)$ 
    Очевидно, что если мощность нашего множества $n$, то $\sigma^n$ = $\mathrm{id} \Longrightarrow 
    \Longrightarrow i = \sigma^{n-k}(j)$
    3) $i \sim j, j \sim e$, то есть $j = \sigma^s(i), \ e = \sigma^t(j) \Longrightarrow e =\sigma^{s+t}(i) \Longrightarrow
    \Longrightarrow$ $\sim$ - эквивалентность $\Longrightarrow$ $\math{X} = \bigcup\limits_{i} \math{X_i}$
    $\sigma \bigg|_{\math{X_i}} = \sigma_i$ - цикл $\Longrightarrow \sigma$ можно записать в виде произведения. $\qed$  \\ \\
     \underline{\textbf{Опр.}} Циклы длины 2 называются \textbf{транспозициями}:
    \begin{center}
        $\sigma_{ij} = \begin{pmatrix}
            i & j \\ j & i
        \end{pmatrix}$
    \end{center}
    \\
    \underline{\textbf{Следствие.}} $\forall \sigma \in S_n$ раскладывается в произведения транспозиций.\\
    
    \textbf{Доказательство:}
    Возьмём цикл длины $k$:
    \begin{pmatrix}
        i_1 & i_2 & ... & i_k \\ i_2 & i_3 & ... & i_1
    \end{pmatrix} = \begin{pmatrix}
        i_1 & i_k \\ i_k & i_1
    \end{pmatrix} = \begin{pmatrix}
        i_1 & i_{k-1} \\ i_{k-1} & i_1
    \end{pmatrix} $...$ \begin{pmatrix}
        i_1 & i_2 \\ i_2 & i_1
    \end{pmatrix} \qed \\
    \underline{\textbf{Опр.}} Пусть $\sigma = \tau_1 ... \tau_k$, где $\tau_i$ - транспозиция. Тогда \textbf{знак перестановки} определяется как:
    \begin{center}
        $\sigma\coloneqq (-1)^k$
    \end{center}
    \newpage
    \underline{\textbf{Теорема.}} 
    \vspace{0.5em}
    $\sigma\in S_n$, тогда:
    1) $\varepsilon_{\sigma}$ не зависит от способа разложения $\sigma$ на траспозиции
    2) $\varepsilon_{\sigma_1 \sigma_2} = \varepsilon_\sigma_1 \cdot \varepsilon_\sigma_2$ , где $\sigma_1, \sigma_2\in S_n$ \\
    \underline{\textbf{Опр.}} $\sigma$ называется \textbf{четной перестановкой}, если ее знак равен $+1$, \textbf{нечётной}, если знак равен $-1$.\\
    \underline{\textbf{Опр.}} Множество всех чётных перестановок есть $\mathbf{A_n}$.\\
    \underline{\textbf{Примеры:}}\\
    1) $id\in A_n$
    2) транспозиции нечетны. Также заметим, что $\tau^{(-1)} = \tau$, а тогда \textbf{$\mathbf{A_n}$ - группа}.\\
    \underline{\textbf{NB.}} \ $|S_n| = n!, \quad |A_n| = \frac{n!}{2}$
    \section*{Основы теории чисел. Делимость}
    \underline{\textbf{Опр.}} Говорят, что $b\neq 0$ \textbf{делит} $a$, если $\exists \ q : a = b\cdot q$
    ($b|a$ - $b$ делит $a$, $a\vdots b$ - $a$ делится на $b$) \\
    \underline{\textbf{Свойства:}}
    1) рефлексивность
    2) на $\mathbb{N}$ антисимметрично
    3) транзитивно
    4) $a|b$, $a|c$ \Longrightarrow $a|(b\pm c)$
    5) $a|b$, $a|(b+c)$ \Longrightarrow $a|c$
    6) $a|b$ \Longrightarrow $\forall \ c$ \quad $a\cdot c | b$
    7) $a|b$ \Longrightarrow $\forall \ k \neq 0$ \quad $k\cdot a | k \cdot b$\\
    \underline{\textbf{Теорема. (деление с остатком)}} 
    \begin{center}
            $\forall a \in \mathbb{Z}, \forall b\in \mathbb{Z}_+\quad \exists ! \ q, r: \ 0 \leq r < b$
    \end{center}
    \underline{\textbf{Доказательство:}} 
    1) Сначала покажем существование. Рассмотрим $a - b\cdot q \ (a>0)$. Выберем такое $q$, что $a-b\cdot q \geq 0 $, тогда это будет наименьшее возможное отрицательное по нашему выбору. Получими, что $r = a = b\cdot q \geq 0$. Кроме этого, $r \leq b$ в силу выбора $q$. Тогда $a - b (q+1) < 0 \Longrightarrow$  $b (q+1) > a \Longrightarrow$
     $ \Longrightarrow r = a - b\cdot q \leq b (q+1) - b\cdot q \leq b$ \Longrightarrow \ $0 \leq r < b$
     \newpage
    2) Покажем единственность. Предположим, что есть $a = b\cdot q_1 + r_1 = b\cdot q_2 + r_2$, \quad $0 \leq r_1, r_2 < b$ \Longrightarrow $|r_1 - r_2| < b $. Также, приравняв $a$, получим $r_1 - r_2 = b (q_2 - q_1)$. Если $q_1, q_2$ различны \Longrightarrow $|r_1 - r_2| \geq b$ $\Longrightarrow$ противоречение 
    $\Longrightarrow q_1 = q_2 \Longrightarrow r_1 = r_2$\\ \qed \\
    \section*{Простые числа}
    \underline{\textbf{Опр.}} $p$ - \textbf{простое}, если $p > 1$ и делится только на $1$ и на $p$.\\
    \underline{\textbf{Опр.}} $a$ - \textbf{составное}, если $a > 1$ и $a = b\cdot c$, где \quad $1<b,c<a$.\\
    \underline{\textbf{NB:}} \mathbb{N} = $\{1\}\cup \{\text{простые}\}\cup \{\text{составные}\}$\\
    \underline{\textbf{Теорема.}}
    \vspace{0.5em}
    $p|a$, $p \neq 1$ и $p$ - наименьший делитель $a$ $\Longrightarrow$ $p$ - простое.\\
    \textbf{Доказательство:}
    $M = \{d\in\mathbb{N} \big| \ d\neq 1 \wedge d|a \}$, \quad $M\neq\varnothing$, так как $a\in M$. Очевидно $M$ ограничено снизу, тогда выберем $p\coloneqq min(M)$. $\existNoBar\ p $ составное \Longrightarrow $p = b\cdot c$
    	\end{obeylines}
    \begin{equation*}
    \begin{cases}
        b < p \\
        c < p
    \end{cases}
        \Longrightarrow c|a \wedge b|a, \text{противоречие, так как} p \neq min(M)
    \end{equation*}
    \qed\\
\begin{obeylines}
\underline{\textbf{Теорема.}} 
\vspace{0.5em}
$\existNoBar \ p$ - наименьший делитель $n$ и $p\neq 1$ \Longrightarrow $p \leq \sqrt{n}$\\
\textbf{Доказательство:}
$n = p\cdot m$, $p \leq m \Longrightarrow n\cdot p\leq n\cdot m \Longrightarrow p^2 \cdot m \leq n\cdot m \Longrightarrow p^2 \leq n$\\ \qed\\
\underline{\textbf{Теорема(Евклида, о бесконечности простых чисел}}
\vspace{0.5em}
\textbf{Доказательство:}\\
От противного: $\existNoBar \ $ множество простых чисел конечно ($n$ штук).
$\existNoBar \ m = p_1\cdot\dots\cdot p_n + 1$ не делится ни на одно простое число $\Longrightarrow m$ - простое, это противоречие.\\ \qed\\
\newpage
\section*{Наибольший общий делитель (gcd)}
\underline{\textbf{Опр.}} \textbf{Наибольшим общим делителем} $a_1, a_2, \dots, a_n$ называется такое число $d>0$:
1) $d|a_i \quad \forall\i$
\vspace{0.2em}
2) $d'|a_i \Longrightarrow d'|d$\\
\underline{\textbf{Опр.}} Числа $a_1, a_2, \dots, a_n$ называются \textbf{взаимно простыми}, если:
\begin{center}
   $gcd(a_1,\dots,a_n) = 1$
\end{center}
\underline{\textbf{Свойства:}}
\vspace{0.2em}
1) $b|a \Longrightarrow gcd(a,b) = b$
\textbf{Доказательство:}
Рассмотрим множества делителей $a$ и $b$. Очевидно, что их пересечение равно $b$, то есть
\begin{center}
    \text{\{делители а и b\}} = \text{\{делители b\}}
\end{center}\\
2) $a = b\cdot q + c \Longrightarrow gcd(a,b) = gcd(b,c)$
3) Алгоритм Евклида
\end{obeylines}
 
\end{document}
